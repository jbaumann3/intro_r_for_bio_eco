% Options for packages loaded elsewhere
\PassOptionsToPackage{unicode}{hyperref}
\PassOptionsToPackage{hyphens}{url}
\PassOptionsToPackage{dvipsnames,svgnames,x11names}{xcolor}
%
\documentclass[
  letterpaper,
  DIV=11,
  numbers=noendperiod]{scrartcl}

\usepackage{amsmath,amssymb}
\usepackage{iftex}
\ifPDFTeX
  \usepackage[T1]{fontenc}
  \usepackage[utf8]{inputenc}
  \usepackage{textcomp} % provide euro and other symbols
\else % if luatex or xetex
  \usepackage{unicode-math}
  \defaultfontfeatures{Scale=MatchLowercase}
  \defaultfontfeatures[\rmfamily]{Ligatures=TeX,Scale=1}
\fi
\usepackage{lmodern}
\ifPDFTeX\else  
    % xetex/luatex font selection
\fi
% Use upquote if available, for straight quotes in verbatim environments
\IfFileExists{upquote.sty}{\usepackage{upquote}}{}
\IfFileExists{microtype.sty}{% use microtype if available
  \usepackage[]{microtype}
  \UseMicrotypeSet[protrusion]{basicmath} % disable protrusion for tt fonts
}{}
\makeatletter
\@ifundefined{KOMAClassName}{% if non-KOMA class
  \IfFileExists{parskip.sty}{%
    \usepackage{parskip}
  }{% else
    \setlength{\parindent}{0pt}
    \setlength{\parskip}{6pt plus 2pt minus 1pt}}
}{% if KOMA class
  \KOMAoptions{parskip=half}}
\makeatother
\usepackage{xcolor}
\setlength{\emergencystretch}{3em} % prevent overfull lines
\setcounter{secnumdepth}{5}
% Make \paragraph and \subparagraph free-standing
\ifx\paragraph\undefined\else
  \let\oldparagraph\paragraph
  \renewcommand{\paragraph}[1]{\oldparagraph{#1}\mbox{}}
\fi
\ifx\subparagraph\undefined\else
  \let\oldsubparagraph\subparagraph
  \renewcommand{\subparagraph}[1]{\oldsubparagraph{#1}\mbox{}}
\fi

\usepackage{color}
\usepackage{fancyvrb}
\newcommand{\VerbBar}{|}
\newcommand{\VERB}{\Verb[commandchars=\\\{\}]}
\DefineVerbatimEnvironment{Highlighting}{Verbatim}{commandchars=\\\{\}}
% Add ',fontsize=\small' for more characters per line
\usepackage{framed}
\definecolor{shadecolor}{RGB}{241,243,245}
\newenvironment{Shaded}{\begin{snugshade}}{\end{snugshade}}
\newcommand{\AlertTok}[1]{\textcolor[rgb]{0.68,0.00,0.00}{#1}}
\newcommand{\AnnotationTok}[1]{\textcolor[rgb]{0.37,0.37,0.37}{#1}}
\newcommand{\AttributeTok}[1]{\textcolor[rgb]{0.40,0.45,0.13}{#1}}
\newcommand{\BaseNTok}[1]{\textcolor[rgb]{0.68,0.00,0.00}{#1}}
\newcommand{\BuiltInTok}[1]{\textcolor[rgb]{0.00,0.23,0.31}{#1}}
\newcommand{\CharTok}[1]{\textcolor[rgb]{0.13,0.47,0.30}{#1}}
\newcommand{\CommentTok}[1]{\textcolor[rgb]{0.37,0.37,0.37}{#1}}
\newcommand{\CommentVarTok}[1]{\textcolor[rgb]{0.37,0.37,0.37}{\textit{#1}}}
\newcommand{\ConstantTok}[1]{\textcolor[rgb]{0.56,0.35,0.01}{#1}}
\newcommand{\ControlFlowTok}[1]{\textcolor[rgb]{0.00,0.23,0.31}{#1}}
\newcommand{\DataTypeTok}[1]{\textcolor[rgb]{0.68,0.00,0.00}{#1}}
\newcommand{\DecValTok}[1]{\textcolor[rgb]{0.68,0.00,0.00}{#1}}
\newcommand{\DocumentationTok}[1]{\textcolor[rgb]{0.37,0.37,0.37}{\textit{#1}}}
\newcommand{\ErrorTok}[1]{\textcolor[rgb]{0.68,0.00,0.00}{#1}}
\newcommand{\ExtensionTok}[1]{\textcolor[rgb]{0.00,0.23,0.31}{#1}}
\newcommand{\FloatTok}[1]{\textcolor[rgb]{0.68,0.00,0.00}{#1}}
\newcommand{\FunctionTok}[1]{\textcolor[rgb]{0.28,0.35,0.67}{#1}}
\newcommand{\ImportTok}[1]{\textcolor[rgb]{0.00,0.46,0.62}{#1}}
\newcommand{\InformationTok}[1]{\textcolor[rgb]{0.37,0.37,0.37}{#1}}
\newcommand{\KeywordTok}[1]{\textcolor[rgb]{0.00,0.23,0.31}{#1}}
\newcommand{\NormalTok}[1]{\textcolor[rgb]{0.00,0.23,0.31}{#1}}
\newcommand{\OperatorTok}[1]{\textcolor[rgb]{0.37,0.37,0.37}{#1}}
\newcommand{\OtherTok}[1]{\textcolor[rgb]{0.00,0.23,0.31}{#1}}
\newcommand{\PreprocessorTok}[1]{\textcolor[rgb]{0.68,0.00,0.00}{#1}}
\newcommand{\RegionMarkerTok}[1]{\textcolor[rgb]{0.00,0.23,0.31}{#1}}
\newcommand{\SpecialCharTok}[1]{\textcolor[rgb]{0.37,0.37,0.37}{#1}}
\newcommand{\SpecialStringTok}[1]{\textcolor[rgb]{0.13,0.47,0.30}{#1}}
\newcommand{\StringTok}[1]{\textcolor[rgb]{0.13,0.47,0.30}{#1}}
\newcommand{\VariableTok}[1]{\textcolor[rgb]{0.07,0.07,0.07}{#1}}
\newcommand{\VerbatimStringTok}[1]{\textcolor[rgb]{0.13,0.47,0.30}{#1}}
\newcommand{\WarningTok}[1]{\textcolor[rgb]{0.37,0.37,0.37}{\textit{#1}}}

\providecommand{\tightlist}{%
  \setlength{\itemsep}{0pt}\setlength{\parskip}{0pt}}\usepackage{longtable,booktabs,array}
\usepackage{calc} % for calculating minipage widths
% Correct order of tables after \paragraph or \subparagraph
\usepackage{etoolbox}
\makeatletter
\patchcmd\longtable{\par}{\if@noskipsec\mbox{}\fi\par}{}{}
\makeatother
% Allow footnotes in longtable head/foot
\IfFileExists{footnotehyper.sty}{\usepackage{footnotehyper}}{\usepackage{footnote}}
\makesavenoteenv{longtable}
\usepackage{graphicx}
\makeatletter
\def\maxwidth{\ifdim\Gin@nat@width>\linewidth\linewidth\else\Gin@nat@width\fi}
\def\maxheight{\ifdim\Gin@nat@height>\textheight\textheight\else\Gin@nat@height\fi}
\makeatother
% Scale images if necessary, so that they will not overflow the page
% margins by default, and it is still possible to overwrite the defaults
% using explicit options in \includegraphics[width, height, ...]{}
\setkeys{Gin}{width=\maxwidth,height=\maxheight,keepaspectratio}
% Set default figure placement to htbp
\makeatletter
\def\fps@figure{htbp}
\makeatother

\KOMAoption{captions}{tableheading}
\makeatletter
\makeatother
\makeatletter
\makeatother
\makeatletter
\@ifpackageloaded{caption}{}{\usepackage{caption}}
\AtBeginDocument{%
\ifdefined\contentsname
  \renewcommand*\contentsname{Table of contents}
\else
  \newcommand\contentsname{Table of contents}
\fi
\ifdefined\listfigurename
  \renewcommand*\listfigurename{List of Figures}
\else
  \newcommand\listfigurename{List of Figures}
\fi
\ifdefined\listtablename
  \renewcommand*\listtablename{List of Tables}
\else
  \newcommand\listtablename{List of Tables}
\fi
\ifdefined\figurename
  \renewcommand*\figurename{Figure}
\else
  \newcommand\figurename{Figure}
\fi
\ifdefined\tablename
  \renewcommand*\tablename{Table}
\else
  \newcommand\tablename{Table}
\fi
}
\@ifpackageloaded{float}{}{\usepackage{float}}
\floatstyle{ruled}
\@ifundefined{c@chapter}{\newfloat{codelisting}{h}{lop}}{\newfloat{codelisting}{h}{lop}[chapter]}
\floatname{codelisting}{Listing}
\newcommand*\listoflistings{\listof{codelisting}{List of Listings}}
\makeatother
\makeatletter
\@ifpackageloaded{caption}{}{\usepackage{caption}}
\@ifpackageloaded{subcaption}{}{\usepackage{subcaption}}
\makeatother
\makeatletter
\@ifpackageloaded{tcolorbox}{}{\usepackage[skins,breakable]{tcolorbox}}
\makeatother
\makeatletter
\@ifundefined{shadecolor}{\definecolor{shadecolor}{rgb}{.97, .97, .97}}
\makeatother
\makeatletter
\makeatother
\makeatletter
\makeatother
\ifLuaTeX
  \usepackage{selnolig}  % disable illegal ligatures
\fi
\IfFileExists{bookmark.sty}{\usepackage{bookmark}}{\usepackage{hyperref}}
\IfFileExists{xurl.sty}{\usepackage{xurl}}{} % add URL line breaks if available
\urlstyle{same} % disable monospaced font for URLs
\hypersetup{
  pdftitle={Packages \& reading/writing data},
  pdfauthor={Justin Baumann},
  colorlinks=true,
  linkcolor={blue},
  filecolor={Maroon},
  citecolor={Blue},
  urlcolor={Blue},
  pdfcreator={LaTeX via pandoc}}

\title{Packages \& reading/writing data}
\author{Justin Baumann}
\date{}

\begin{document}
\maketitle
\ifdefined\Shaded\renewenvironment{Shaded}{\begin{tcolorbox}[frame hidden, enhanced, breakable, sharp corners, borderline west={3pt}{0pt}{shadecolor}, interior hidden, boxrule=0pt]}{\end{tcolorbox}}\fi

\renewcommand*\contentsname{Table of contents}
{
\hypersetup{linkcolor=}
\setcounter{tocdepth}{3}
\tableofcontents
}
\hypertarget{learning-objectives}{%
\section{\texorpdfstring{\textbf{Learning
Objectives}}{Learning Objectives}}\label{learning-objectives}}

1.) How to install and load libraries\\
2.) How to view and inspect data\\
3.) Read in data files \& output data (write to file)\\

\hypertarget{r-basics}{%
\section{\texorpdfstring{\textbf{R Basics}}{R Basics}}\label{r-basics}}

\hypertarget{installing-and-loading-libraries}{%
\subsection{\texorpdfstring{\textbf{Installing and Loading
libraries}}{Installing and Loading libraries}}\label{installing-and-loading-libraries}}

Libraries are packages of functions (and sometimes data) that we use to
execute tasks in R. Packages are what make R so versatile! We can do
almost anything with R if we learn how to utilize the right packages.

If we do not have a package already installed (for example, if you have
only just downloaded R/ RStudio), we will need to use
install.packages(`packagename') to install each package that we need.

\begin{Shaded}
\begin{Highlighting}[]
\FunctionTok{install.packages}\NormalTok{(tidyverse)}
\end{Highlighting}
\end{Shaded}

OR - We can use the `Packages' tab in the bottom right quadrant to
install packages. Simply navigate to `Packages', select `install
packages' and enter the package names you need (separate each package by
commas). \textbf{NOTE} for users for rstudio.mtholyoke.edu -- You cannot
install packages to the Mt Holyoke cloud instance of R. If we need
something that isn't installed we will need to contact IT!

In order for a \emph{package to work}, we must first load it! We do this
as with the code libary(packagename)

\begin{Shaded}
\begin{Highlighting}[]
\FunctionTok{library}\NormalTok{(tidyverse) }\CommentTok{\#for data manipulation}
\FunctionTok{library}\NormalTok{(palmerpenguins) }\CommentTok{\#for some fun data!}
\end{Highlighting}
\end{Shaded}

\textbf{It is best practice to load all of the packages you will need at
the top of your script}

In this course we will be following a best practices guide that utilizes
a library called `Tidyverse' for data manipulation and analysis.
Tidyverse contains many packages all in one, including the very
functional `dplyr' and `ggplot2' packages. You will almost always use
Tidyverse, so make sure to load it in :)

Note the `\#' with notes after them in the code chunk above. These are
called comments. You can comment out any line of code in R by using a
`\#'. This is strongly recommended when you are programming. We will
discuss more later!

\hypertarget{looking-at-data}{%
\subsection{\texorpdfstring{\textbf{Looking at
data!}}{Looking at data!}}\label{looking-at-data}}

R has integrated data sets that we can use to play around with code and
learn.

examples: mtcars (a dataframe all about cars, this is available in R
without loading a package), and iris (in the `vegan' package, great for
testing out ecology related functions and code)

\textbf{Load a dataset} R has some test datasets built into it. Let's
load one and look at it!

\begin{Shaded}
\begin{Highlighting}[]
\NormalTok{mtcars }
\end{Highlighting}
\end{Shaded}

\begin{verbatim}
                     mpg cyl  disp  hp drat    wt  qsec vs am gear carb
Mazda RX4           21.0   6 160.0 110 3.90 2.620 16.46  0  1    4    4
Mazda RX4 Wag       21.0   6 160.0 110 3.90 2.875 17.02  0  1    4    4
Datsun 710          22.8   4 108.0  93 3.85 2.320 18.61  1  1    4    1
Hornet 4 Drive      21.4   6 258.0 110 3.08 3.215 19.44  1  0    3    1
Hornet Sportabout   18.7   8 360.0 175 3.15 3.440 17.02  0  0    3    2
Valiant             18.1   6 225.0 105 2.76 3.460 20.22  1  0    3    1
Duster 360          14.3   8 360.0 245 3.21 3.570 15.84  0  0    3    4
Merc 240D           24.4   4 146.7  62 3.69 3.190 20.00  1  0    4    2
Merc 230            22.8   4 140.8  95 3.92 3.150 22.90  1  0    4    2
Merc 280            19.2   6 167.6 123 3.92 3.440 18.30  1  0    4    4
Merc 280C           17.8   6 167.6 123 3.92 3.440 18.90  1  0    4    4
Merc 450SE          16.4   8 275.8 180 3.07 4.070 17.40  0  0    3    3
Merc 450SL          17.3   8 275.8 180 3.07 3.730 17.60  0  0    3    3
Merc 450SLC         15.2   8 275.8 180 3.07 3.780 18.00  0  0    3    3
Cadillac Fleetwood  10.4   8 472.0 205 2.93 5.250 17.98  0  0    3    4
Lincoln Continental 10.4   8 460.0 215 3.00 5.424 17.82  0  0    3    4
Chrysler Imperial   14.7   8 440.0 230 3.23 5.345 17.42  0  0    3    4
Fiat 128            32.4   4  78.7  66 4.08 2.200 19.47  1  1    4    1
Honda Civic         30.4   4  75.7  52 4.93 1.615 18.52  1  1    4    2
Toyota Corolla      33.9   4  71.1  65 4.22 1.835 19.90  1  1    4    1
Toyota Corona       21.5   4 120.1  97 3.70 2.465 20.01  1  0    3    1
Dodge Challenger    15.5   8 318.0 150 2.76 3.520 16.87  0  0    3    2
AMC Javelin         15.2   8 304.0 150 3.15 3.435 17.30  0  0    3    2
Camaro Z28          13.3   8 350.0 245 3.73 3.840 15.41  0  0    3    4
Pontiac Firebird    19.2   8 400.0 175 3.08 3.845 17.05  0  0    3    2
Fiat X1-9           27.3   4  79.0  66 4.08 1.935 18.90  1  1    4    1
Porsche 914-2       26.0   4 120.3  91 4.43 2.140 16.70  0  1    5    2
Lotus Europa        30.4   4  95.1 113 3.77 1.513 16.90  1  1    5    2
Ford Pantera L      15.8   8 351.0 264 4.22 3.170 14.50  0  1    5    4
Ferrari Dino        19.7   6 145.0 175 3.62 2.770 15.50  0  1    5    6
Maserati Bora       15.0   8 301.0 335 3.54 3.570 14.60  0  1    5    8
Volvo 142E          21.4   4 121.0 109 4.11 2.780 18.60  1  1    4    2
\end{verbatim}

\textbf{Using head() and tail()} Now let's look at the data frame (df)
using head() and tail() These tell us the column names, and let us see
the top or bottom 6 rows of data.

\begin{Shaded}
\begin{Highlighting}[]
\FunctionTok{head}\NormalTok{(mtcars) }
\end{Highlighting}
\end{Shaded}

\begin{verbatim}
                   mpg cyl disp  hp drat    wt  qsec vs am gear carb
Mazda RX4         21.0   6  160 110 3.90 2.620 16.46  0  1    4    4
Mazda RX4 Wag     21.0   6  160 110 3.90 2.875 17.02  0  1    4    4
Datsun 710        22.8   4  108  93 3.85 2.320 18.61  1  1    4    1
Hornet 4 Drive    21.4   6  258 110 3.08 3.215 19.44  1  0    3    1
Hornet Sportabout 18.7   8  360 175 3.15 3.440 17.02  0  0    3    2
Valiant           18.1   6  225 105 2.76 3.460 20.22  1  0    3    1
\end{verbatim}

\begin{Shaded}
\begin{Highlighting}[]
\FunctionTok{tail}\NormalTok{(mtcars) }\CommentTok{\#tail shows the header and the last 6 rows }
\end{Highlighting}
\end{Shaded}

\begin{verbatim}
                mpg cyl  disp  hp drat    wt qsec vs am gear carb
Porsche 914-2  26.0   4 120.3  91 4.43 2.140 16.7  0  1    5    2
Lotus Europa   30.4   4  95.1 113 3.77 1.513 16.9  1  1    5    2
Ford Pantera L 15.8   8 351.0 264 4.22 3.170 14.5  0  1    5    4
Ferrari Dino   19.7   6 145.0 175 3.62 2.770 15.5  0  1    5    6
Maserati Bora  15.0   8 301.0 335 3.54 3.570 14.6  0  1    5    8
Volvo 142E     21.4   4 121.0 109 4.11 2.780 18.6  1  1    4    2
\end{verbatim}

\textbf{column attributes} If we want to see the attributes of each
column we can use the str() function

\begin{Shaded}
\begin{Highlighting}[]
\FunctionTok{str}\NormalTok{(mtcars) }\CommentTok{\#str shows attributes of each column}
\end{Highlighting}
\end{Shaded}

\begin{verbatim}
'data.frame':   32 obs. of  11 variables:
 $ mpg : num  21 21 22.8 21.4 18.7 18.1 14.3 24.4 22.8 19.2 ...
 $ cyl : num  6 6 4 6 8 6 8 4 4 6 ...
 $ disp: num  160 160 108 258 360 ...
 $ hp  : num  110 110 93 110 175 105 245 62 95 123 ...
 $ drat: num  3.9 3.9 3.85 3.08 3.15 2.76 3.21 3.69 3.92 3.92 ...
 $ wt  : num  2.62 2.88 2.32 3.21 3.44 ...
 $ qsec: num  16.5 17 18.6 19.4 17 ...
 $ vs  : num  0 0 1 1 0 1 0 1 1 1 ...
 $ am  : num  1 1 1 0 0 0 0 0 0 0 ...
 $ gear: num  4 4 4 3 3 3 3 4 4 4 ...
 $ carb: num  4 4 1 1 2 1 4 2 2 4 ...
\end{verbatim}

str() is very important because it allows you to see the type of data in
each column. Types include: integer, numeric, factor, date, and more. If
the data in a column are factors instead of numbers you may have an
issue in your data (your spreadsheet)

\textbf{Changing column attributes} Importantly, you can change the type
of the column. Here is an example

\begin{Shaded}
\begin{Highlighting}[]
\NormalTok{mtcars}\SpecialCharTok{$}\NormalTok{mpg}\OtherTok{=}\FunctionTok{as.factor}\NormalTok{(mtcars}\SpecialCharTok{$}\NormalTok{mpg) }\CommentTok{\# Makes mpg a factor instead of a number}
\FunctionTok{str}\NormalTok{(mtcars)}
\end{Highlighting}
\end{Shaded}

\begin{verbatim}
'data.frame':   32 obs. of  11 variables:
 $ mpg : Factor w/ 25 levels "10.4","13.3",..: 16 16 19 17 13 12 3 20 19 14 ...
 $ cyl : num  6 6 4 6 8 6 8 4 4 6 ...
 $ disp: num  160 160 108 258 360 ...
 $ hp  : num  110 110 93 110 175 105 245 62 95 123 ...
 $ drat: num  3.9 3.9 3.85 3.08 3.15 2.76 3.21 3.69 3.92 3.92 ...
 $ wt  : num  2.62 2.88 2.32 3.21 3.44 ...
 $ qsec: num  16.5 17 18.6 19.4 17 ...
 $ vs  : num  0 0 1 1 0 1 0 1 1 1 ...
 $ am  : num  1 1 1 0 0 0 0 0 0 0 ...
 $ gear: num  4 4 4 3 3 3 3 4 4 4 ...
 $ carb: num  4 4 1 1 2 1 4 2 2 4 ...
\end{verbatim}

\begin{Shaded}
\begin{Highlighting}[]
\NormalTok{mtcars}\SpecialCharTok{$}\NormalTok{mpg}\OtherTok{=}\FunctionTok{as.numeric}\NormalTok{(mtcars}\SpecialCharTok{$}\NormalTok{mpg) }\CommentTok{\#Changes mpg back to a number}
\FunctionTok{str}\NormalTok{(mtcars)}
\end{Highlighting}
\end{Shaded}

\begin{verbatim}
'data.frame':   32 obs. of  11 variables:
 $ mpg : num  16 16 19 17 13 12 3 20 19 14 ...
 $ cyl : num  6 6 4 6 8 6 8 4 4 6 ...
 $ disp: num  160 160 108 258 360 ...
 $ hp  : num  110 110 93 110 175 105 245 62 95 123 ...
 $ drat: num  3.9 3.9 3.85 3.08 3.15 2.76 3.21 3.69 3.92 3.92 ...
 $ wt  : num  2.62 2.88 2.32 3.21 3.44 ...
 $ qsec: num  16.5 17 18.6 19.4 17 ...
 $ vs  : num  0 0 1 1 0 1 0 1 1 1 ...
 $ am  : num  1 1 1 0 0 0 0 0 0 0 ...
 $ gear: num  4 4 4 3 3 3 3 4 4 4 ...
 $ carb: num  4 4 1 1 2 1 4 2 2 4 ...
\end{verbatim}

\textbf{Summary statistics} To see summary statistics on each column
(mean, median, min, max, range), we can use summary()

\begin{Shaded}
\begin{Highlighting}[]
\FunctionTok{summary}\NormalTok{(mtcars) }\CommentTok{\#summarizes each column}
\end{Highlighting}
\end{Shaded}

\begin{verbatim}
      mpg             cyl             disp             hp       
 Min.   : 1.00   Min.   :4.000   Min.   : 71.1   Min.   : 52.0  
 1st Qu.: 6.75   1st Qu.:4.000   1st Qu.:120.8   1st Qu.: 96.5  
 Median :14.00   Median :6.000   Median :196.3   Median :123.0  
 Mean   :13.16   Mean   :6.188   Mean   :230.7   Mean   :146.7  
 3rd Qu.:19.00   3rd Qu.:8.000   3rd Qu.:326.0   3rd Qu.:180.0  
 Max.   :25.00   Max.   :8.000   Max.   :472.0   Max.   :335.0  
      drat             wt             qsec             vs        
 Min.   :2.760   Min.   :1.513   Min.   :14.50   Min.   :0.0000  
 1st Qu.:3.080   1st Qu.:2.581   1st Qu.:16.89   1st Qu.:0.0000  
 Median :3.695   Median :3.325   Median :17.71   Median :0.0000  
 Mean   :3.597   Mean   :3.217   Mean   :17.85   Mean   :0.4375  
 3rd Qu.:3.920   3rd Qu.:3.610   3rd Qu.:18.90   3rd Qu.:1.0000  
 Max.   :4.930   Max.   :5.424   Max.   :22.90   Max.   :1.0000  
       am              gear            carb      
 Min.   :0.0000   Min.   :3.000   Min.   :1.000  
 1st Qu.:0.0000   1st Qu.:3.000   1st Qu.:2.000  
 Median :0.0000   Median :4.000   Median :2.000  
 Mean   :0.4062   Mean   :3.688   Mean   :2.812  
 3rd Qu.:1.0000   3rd Qu.:4.000   3rd Qu.:4.000  
 Max.   :1.0000   Max.   :5.000   Max.   :8.000  
\end{verbatim}

\textbf{Counting rows and columns} To see the number of rows and columns
we can use nrow() and ncol()

\begin{Shaded}
\begin{Highlighting}[]
\FunctionTok{nrow}\NormalTok{(mtcars) }\CommentTok{\#gives number of rows}
\end{Highlighting}
\end{Shaded}

\begin{verbatim}
[1] 32
\end{verbatim}

\begin{Shaded}
\begin{Highlighting}[]
\FunctionTok{ncol}\NormalTok{(mtcars) }\CommentTok{\#gives number of columns}
\end{Highlighting}
\end{Shaded}

\begin{verbatim}
[1] 11
\end{verbatim}

\textbf{Naming dataframes} Rename mtcars and view in Environment tab in
Rstudio

\begin{Shaded}
\begin{Highlighting}[]
\NormalTok{a}\OtherTok{\textless{}{-}}\NormalTok{mtcars}
\NormalTok{a}
\end{Highlighting}
\end{Shaded}

\begin{verbatim}
                    mpg cyl  disp  hp drat    wt  qsec vs am gear carb
Mazda RX4            16   6 160.0 110 3.90 2.620 16.46  0  1    4    4
Mazda RX4 Wag        16   6 160.0 110 3.90 2.875 17.02  0  1    4    4
Datsun 710           19   4 108.0  93 3.85 2.320 18.61  1  1    4    1
Hornet 4 Drive       17   6 258.0 110 3.08 3.215 19.44  1  0    3    1
Hornet Sportabout    13   8 360.0 175 3.15 3.440 17.02  0  0    3    2
Valiant              12   6 225.0 105 2.76 3.460 20.22  1  0    3    1
Duster 360            3   8 360.0 245 3.21 3.570 15.84  0  0    3    4
Merc 240D            20   4 146.7  62 3.69 3.190 20.00  1  0    4    2
Merc 230             19   4 140.8  95 3.92 3.150 22.90  1  0    4    2
Merc 280             14   6 167.6 123 3.92 3.440 18.30  1  0    4    4
Merc 280C            11   6 167.6 123 3.92 3.440 18.90  1  0    4    4
Merc 450SE            9   8 275.8 180 3.07 4.070 17.40  0  0    3    3
Merc 450SL           10   8 275.8 180 3.07 3.730 17.60  0  0    3    3
Merc 450SLC           6   8 275.8 180 3.07 3.780 18.00  0  0    3    3
Cadillac Fleetwood    1   8 472.0 205 2.93 5.250 17.98  0  0    3    4
Lincoln Continental   1   8 460.0 215 3.00 5.424 17.82  0  0    3    4
Chrysler Imperial     4   8 440.0 230 3.23 5.345 17.42  0  0    3    4
Fiat 128             24   4  78.7  66 4.08 2.200 19.47  1  1    4    1
Honda Civic          23   4  75.7  52 4.93 1.615 18.52  1  1    4    2
Toyota Corolla       25   4  71.1  65 4.22 1.835 19.90  1  1    4    1
Toyota Corona        18   4 120.1  97 3.70 2.465 20.01  1  0    3    1
Dodge Challenger      7   8 318.0 150 2.76 3.520 16.87  0  0    3    2
AMC Javelin           6   8 304.0 150 3.15 3.435 17.30  0  0    3    2
Camaro Z28            2   8 350.0 245 3.73 3.840 15.41  0  0    3    4
Pontiac Firebird     14   8 400.0 175 3.08 3.845 17.05  0  0    3    2
Fiat X1-9            22   4  79.0  66 4.08 1.935 18.90  1  1    4    1
Porsche 914-2        21   4 120.3  91 4.43 2.140 16.70  0  1    5    2
Lotus Europa         23   4  95.1 113 3.77 1.513 16.90  1  1    5    2
Ford Pantera L        8   8 351.0 264 4.22 3.170 14.50  0  1    5    4
Ferrari Dino         15   6 145.0 175 3.62 2.770 15.50  0  1    5    6
Maserati Bora         5   8 301.0 335 3.54 3.570 14.60  0  1    5    8
Volvo 142E           17   4 121.0 109 4.11 2.780 18.60  1  1    4    2
\end{verbatim}

\begin{Shaded}
\begin{Highlighting}[]
\FunctionTok{head}\NormalTok{(a)}
\end{Highlighting}
\end{Shaded}

\begin{verbatim}
                  mpg cyl disp  hp drat    wt  qsec vs am gear carb
Mazda RX4          16   6  160 110 3.90 2.620 16.46  0  1    4    4
Mazda RX4 Wag      16   6  160 110 3.90 2.875 17.02  0  1    4    4
Datsun 710         19   4  108  93 3.85 2.320 18.61  1  1    4    1
Hornet 4 Drive     17   6  258 110 3.08 3.215 19.44  1  0    3    1
Hornet Sportabout  13   8  360 175 3.15 3.440 17.02  0  0    3    2
Valiant            12   6  225 105 2.76 3.460 20.22  1  0    3    1
\end{verbatim}

\hypertarget{write-data-to-file-saving-data}{%
\subsection{\texorpdfstring{\textbf{Write data to file (saving
data)}}{Write data to file (saving data)}}\label{write-data-to-file-saving-data}}

We use the write.csv function here. a= the name of the dataframe and the
name we want to give the file goes after `file=' The file name must be
in quotes and must include an extension. Since we are using write.csv we
MUST use .csv

\begin{Shaded}
\begin{Highlighting}[]
\FunctionTok{write.csv}\NormalTok{(a, }\AttributeTok{file=}\StringTok{\textquotesingle{}mtcars.csv\textquotesingle{}}\NormalTok{)}
\end{Highlighting}
\end{Shaded}

\hypertarget{read-a-file-in-import-data-into-r}{%
\subsection{\texorpdfstring{\textbf{Read a file in (import data into
R)}}{Read a file in (import data into R)}}\label{read-a-file-in-import-data-into-r}}

NOTE: if you have a .xls file make sure you convert to .csv. Ensure the
file is clean and orderly (rows x columns). Only 1 excel tab can be in
each .csv, so plan accordingly

\begin{Shaded}
\begin{Highlighting}[]
\NormalTok{b}\OtherTok{\textless{}{-}}\FunctionTok{read.csv}\NormalTok{(}\StringTok{\textquotesingle{}mtcars.csv\textquotesingle{}}\NormalTok{)}
\FunctionTok{head}\NormalTok{(b)}
\end{Highlighting}
\end{Shaded}

\begin{verbatim}
                  X mpg cyl disp  hp drat    wt  qsec vs am gear carb
1         Mazda RX4  16   6  160 110 3.90 2.620 16.46  0  1    4    4
2     Mazda RX4 Wag  16   6  160 110 3.90 2.875 17.02  0  1    4    4
3        Datsun 710  19   4  108  93 3.85 2.320 18.61  1  1    4    1
4    Hornet 4 Drive  17   6  258 110 3.08 3.215 19.44  1  0    3    1
5 Hornet Sportabout  13   8  360 175 3.15 3.440 17.02  0  0    3    2
6           Valiant  12   6  225 105 2.76 3.460 20.22  1  0    3    1
\end{verbatim}

You are welcome to use other functions to read in data (including
read\_csv or read.xls). Especially for beginners, I strongly encourage
you to use .csv format. Other file formats can get complicated (often
unnecessarily complicated). That said, R can also handle .txt, .xls,
images, shapefiles (for spatial analysis or GIS style work), etc. It is
very versatile! Feel free to explore :)\\
A note on read\_csv -\textgreater{} I consider this to be the ``best''
option for reading in .csv files. It is a `smarter' version of read.csv
and can automatically figure out what kind of data (numeric, factor,
date, etc) each column is. If you use read.csv, you have often have to
manually change these options.

\begin{center}\rule{0.5\linewidth}{0.5pt}\end{center}



\end{document}
